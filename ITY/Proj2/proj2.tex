\documentclass[a4paper, 11pt, twocolumn]{article}
\usepackage[czech]{babel}
\usepackage[utf8]{inputenc}
\usepackage[IL2]{fontenc}
\usepackage[left=1.5cm, top=2.5cm, text={18cm, 25cm}]{geometry}
\usepackage{times}
\usepackage{amsthm, amssymb, amsmath}

\theoremstyle{definition}
\newtheorem{definice}{Definice}

\theoremstyle{definition}
\newtheorem{veta}{Věta}

\begin{document}

\begin{titlepage}
    \begin{center}
    \Huge\textsc{Vysoké učení technické v Brně\\
                \huge{Fakulta informačních technologií}}\\
    \vspace{\stretch{0.382}}
    \LARGE{Typografie a publikování\,--\,2. projekt}\\
            Sazba dokumentů a matematických výrazů\\
    \vspace{\stretch{0.618}}
\end{center}
{\Large 2022 \hfill xxxx (xlogin00)}
\end{titlepage}

\section*{Úvod}
V této úloze si vyzkoušíme sazbu titulní strany, matematických vzorců, prostředí a dalších textových struktur obvyklých pro technicky zaměřené texty (například rovnice (\ref{rov2}) nebo Definice \ref{def2} na straně \pageref{def2}). Pro vytvoření těchto odkazů používáme příkazy \verb|\label, \ref a \pageref|.\par
Na titulní straně je využito sázení nadpisu podle optického středu s využitím zlatého řezu. Tento postup byl probírán na přednášce. Dále je na titulní straně použito odřádkování se zadanou relativní velikostí 0,4em a 0,3em.

\section{Matematický text}
Nejprve se podíváme na sázení matematických symbolů a výrazů v plynulém textu včetně sazby definic a vět s využitím balíku \verb|amsthm|. Rovněž použijeme poznámku pod čarou s použitím příkazu \verb|\footnote|. Někdy je vhodné použít konstrukci \verb|${}$| nebo \verb|\mbox{}|, která říká, že (matematický) text nemá být zalomen. 

\begin{definice} 
     Nedeterministický Turingův stroj (NTS) \emph{je šestice tvaru $ M = (Q, \Sigma, \Gamma, \delta, g_0, q_F)$, kde:}
    \begin{itemize}
        \item $Q$ \emph{je konečná množina} vnitřních (řídicích) stavů,
        \item $\Sigma$ \emph{je konečná množina symbolů nazývaná} vstupní abeceda, $\Delta \notin \Sigma$,
        \item $\Gamma$ \emph{je konečná množina symbolů, $\Sigma \subset \Gamma$, $\Delta \in \Gamma$, nazývaná} pásková abeceda,
        \item $\delta : (Q\setminus\{q_F\} \times \Gamma \rightarrow 2^{Q\times(\Gamma\cup\{L,R\})}$\emph{, kde L, R $\notin \Gamma$,\\ je parciální} přechodová funkce, \emph{a}
        \item $q_0 \in Q$ \emph{je} počáteční stav \emph{a $q_F \in Q$ je} koncový stav.
    \end{itemize}
\end{definice}

Symbol $\Delta$ značí tzv. \emph{blank} (prázdný symbol), který se vyskytuje na místech pásky, která nebyla ještě použita.\par
\emph{Konfigurace pásky} se skládá z nekonečného řetězce, který reprezentuje obsah pásky, a pozice hlavy na tomto řetězci. Jedná se o prvek množiny $ \{\gamma \Delta^\omega \mid \gamma \in \Gamma^*\} \times \mathbb{N}\footnote{Pro libovolnou abecedu $\Sigma$ je $\Sigma^\omega$  množina všech \emph{nekonečných} řetězců nad $\Sigma$, tj. nekonečných posloupností symbolů ze $\Sigma$}$. \emph{Konfigurace pásky} obvykle zapisujeme jako $\Delta xyz\underline{z}x \Delta ...$ (podtržení značí pozici hlavy).\emph{Konfigurace stroje} je pak dána stavem řízení a konfigurací pásky. Formálně se jedná o prvek množiny $ Q \times \{\gamma \Delta^\omega \mid \gamma \in \Gamma^*\} \times \mathbb{N}$.

\subsection{Podsekce obsahující definici a větu}
\begin{definice}\label{def2}
Řetězec $\omega$ nad abecedou $\Sigma$ je přijat NTS \emph{$M$, jestliže $M$ při aktivaci z počáteční konfigurace pásky $\underline{\Delta}\omega\Delta$\dots a počátečního stavu $q_0$ může zastavit přechodem do koncového stavu $q_F$, tj. $(q_0, \Delta\omega\Delta^\omega, 0) \underset{M}{\overset{*}{\vdash}} (q_F,\gamma,n)$ pro nějaké $\gamma \in \Gamma^*$ a} $n \in \mathbb{N}$.\par
\emph{Množinu $L(M) = \{\omega \mid \omega$ je přijat NTS $M\} \subseteq \Sigma^*$  nazýváme} jazyk přijímaný NTS $M$.\par
\setlength{\parskip}{1em}
Nyní si vyzkoušíme sazbu vět a důkazů opět s použitím balíku \verb|amsthm|. 
\end{definice}

\begin{veta}
\emph{Třída jazyků, které jsou přijímány NTS, odpovídá} rekurzivně vyčíslitelným jazykům.
\end{veta}

\section{Rovnice}
Složitější matematické formulace sázíme mimo plynulý text. Lze umístit několik výrazů na jeden řádek, ale pak je třeba tyto vhodně oddělit, například příkazem \verb|\quad|.

$$x^2 - \sqrt[4]{y_1 * y_2^3} \quad  x > y_1 \geq y_2 \quad z_{z_z} \neq \alpha_1^{\alpha_2^{\alpha_3}}$$

V rovnici (\ref{rov1}) jsou využity tři typy závorek s různou explicitně definovanou velikostí.

\begin{eqnarray}
\label{rov1}
    x&=&\bigg\{a \oplus \Big[b \cdot \big(c \ominus d \big)\Big]\bigg\}^{4/2}\\
\label{rov2}
    y&=&\lim_{\beta\to\infty}\frac{\tan^2 \beta - \sin^3 \beta}{\frac{1}{\frac{1}{\log_{42}x}+ \frac{1}{2}}} 
\end{eqnarray}

V této větě vidíme, jak vypadá implicitní vysázení limity $\lim_{n\to\infty}f(n)$ v normálním odstavci textu. Podobně je to i s dalšími symboly jako $\bigcup_{N \in \mathcal{M}}N$ či $\sum_{j=0}^n x_j^2$
S vynucením méně úsporné sazby příkazem \verb|\limits| budou vzorce vysázeny v podobě $\lim\limits_{n\to\infty}f(n)$ a  $\sum\limits_{j=0}^n x_j^2$ . 

\section{Matice}
Pro sázení matic se velmi často používá prostředí \verb|array| a závorky (\verb|\left, \right|). 

$$\mathbf{A} = \left|\begin{array}{cccc}
     a_{11}&a_{12}&\dots&a_{1n}\\
     a_{21}&a_{22}&\dots&a_{2n}\\
     \vdots&\vdots&\ddots&\vdots\\
     a_{m1}&a_{m2}&\dots&a_{mn}\\
\end{array} \right| = 
\left|\begin{array}{cc}
     t&u\\
     u&w\\ 
\end{array} \right |= tw - uv$$

Prostředí \verb|array| lze úspěšně využít i jinde.

$$\binom{n}{k}=
\begin{cases}
    \frac{n!}{k!(n-k)!}&$pro $0\leq k \leq n\\
    0&$pro $k>n$ nebo $k<0\\
\end{cases}$$

\end{document}
