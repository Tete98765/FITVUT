\documentclass[10pt]{beamer}
\usepackage[utf8]{inputenc}
\usepackage[IL2]{fontenc}
\usepackage{times}
\usepackage{graphics}
\usepackage{listings}
\usepackage{color}
\usepackage[slovak]{babel}
\usepackage{url}
\usepackage[czech, linesnumbered, noline, ruled, longend]{algorithm2e}

\usetheme{CambridgeUS}

\title{Quick Sort}
\author{xxxxx(xlogin00)}
\date{\today}

\begin{document}
\maketitle

\begin{frame}{Quick Sort}
    \begin{itemize}
        \item alebo \textbf{rýchle triedenie} je veľmi rýchly nestabilný triediaci algoritmus
        \item jeho priemerná doba výpočtu je $O(n.\log(n))$ 
        \item pri nevhodnom tvare vstupných dát môže jeho zložitosť byť $O(n^2)$ (napríklad pri triedení už zotriedeného poľa)
    \end{itemize}
\end{frame}

\begin{frame}{Quicksort - princíp}
    \begin{itemize}
        \item vstupnú postupnosť rozdelí na dve približne rovnaké časti
            \begin{itemize}
                \item zvolíme si pivot postupnosti
                \item na ľavo by mali byť prvky menšie než pivot
                \item na pravo by mali byť prvky väčšie než pivot
            \end{itemize} 
        \item na rozdelenie postupnosti sa používa procedúra \textbf{partition}, ktorá sa zavolá po tom, ako si zvolíme pivota
    \end{itemize}
\end{frame}

\begin{frame}{Quicksort - princíp}
    \begin{itemize}
        \item zvolíme si pivota postupnosti
    
    \begin{figure}[h] 
        \centering
        \setlength{\unitlength}{0.5cm}
        \begin{picture}(10,6.5)
            \put(0,6){\line(1,0){7}}
            \put(0,6){\line(0,-1){1}}
            \put(1,6){\line(0,-1){1}}
            \put(2,6){\line(0,-1){1}}
            \put(3,6){\line(0,-1){1}}
            \put(4,6){\line(0,-1){1}}
            \put(5,6){\line(0,-1){1}}
            \put(6,6){\line(0,-1){1}}
            \put(7,6){\line(0,-1){1}}
            \put(0,5){\line(1,0){7}}
            
            \put(0.3,5.3){8}
            \put(1.3,5.3){7}
            \put(2.3,5.3){6}
            \put(3.3,5.3){1}
            \put(4.3,5.3){0}
            \put(5.3,5.3){9}
            \put(6.3,5.3){\color{red}{2}}
            \end{picture}
            
        \end{figure}
        \pause
        
        \item prvky poľa premiestníme tak, aby tie, ktoré sú menšie, boli naľavo od pivota a tie, ktoré sú väčšie na pravo
        
        \begin{figure}[h] 
        \centering
        \setlength{\unitlength}{0.5cm}
        \begin{picture}(10,6.5)
            \put(0,6){\line(1,0){7}}
            \put(0,6){\line(0,-1){1}}
            \put(1,6){\line(0,-1){1}}
            \put(2,6){\line(0,-1){1}}
            \put(3,6){\line(0,-1){1}}
            \put(4,6){\line(0,-1){1}}
            \put(5,6){\line(0,-1){1}}
            \put(6,6){\line(0,-1){1}}
            \put(7,6){\line(0,-1){1}}
            \put(0,5){\line(1,0){7}}
            
            \put(0.3,5.3){1}
            \put(1.3,5.3){0}
            \put(2.3,5.3){\color{red}{2}}
            \put(3.3,5.3){8}
            \put(4.3,5.3){7}
            \put(5.3,5.3){9}
            \put(6.3,5.3){6}
            \end{picture}
        \end{figure}
    \end{itemize}
\end{frame}

\begin{frame}{Partition}
    \begin{itemize}
        \item začíname na prvku úplne naľavo s indexom i
        \item pokiaľ je pivot menší než prvok na indexe i, uložíme si ukazateľ na tento prvok
        \item pivot sa ďalej porovnáva s ostatnými prvkami
        \item ak narazí na prvok menší, než je pivot, tak sa tento prvok prehodí s väčším prvkom, ktorý bol nájdený skôr
        \item proces pokračuje kým nenarazí na predposledný prvok
    \end{itemize}
\end{frame}

\begin{frame}{Partition}
    \begin{figure}[h] 
        \centering
        \setlength{\unitlength}{0.5cm}
        \begin{picture}(20,6.5)
            \put(0,10){\line(1,0){7}} 
            \put(0,10){\line(0,-1){1}} 
            \put(1,10){\line(0,-1){1}}
            \put(2,10){\line(0,-1){1}}
            \put(3,10){\line(0,-1){1}}
            \put(4,10){\line(0,-1){1}}
            \put(5,10){\line(0,-1){1}}
            \put(6,10){\line(0,-1){1}}
            \put(7,10){\line(0,-1){1}}
            \put(0,9){\line(1,0){7}}
            
            \put(6.5,8){\line(0,1){1}}
            \put(0.5,8){\line(1,0){6}}
            \put(0.5,8){\line(0,1){1}}
            
            \put(0.3,9.3){\color{blue}{8}}
            \put(1.3,9.3){7}
            \put(2.3,9.3){6}
            \put(3.3,9.3){1}
            \put(4.3,9.3){0}
            \put(5.3,9.3){9}
            \put(6.3,9.3){\color{red}{2}}
            \put(7.3,9.3){2 porovná s 8}
            \put(7.3,8){8 je väčšia ako dva, preto sa na ňu uloží ukazateľ}
            
        \pause

            \put(0,7){\line(1,0){7}}
            \put(0,7){\line(0,-1){1}}
            \put(1,7){\line(0,-1){1}}
            \put(2,7){\line(0,-1){1}}
            \put(3,7){\line(0,-1){1}}
            \put(4,7){\line(0,-1){1}}
            \put(5,7){\line(0,-1){1}}
            \put(6,7){\line(0,-1){1}}
            \put(7,7){\line(0,-1){1}}
            \put(0,6){\line(1,0){7}}
            
            \put(6.5,5){\line(0,1){1}}
            \put(1.5,5){\line(1,0){5}}
            \put(1.5,5){\line(0,1){1}}
            
            \put(0.3,6.3){\color{blue}{8}}
            \put(1.3,6.3){7}
            \put(2.3,6.3){6}
            \put(3.3,6.3){1}
            \put(4.3,6.3){0}
            \put(5.3,6.3){9}
            \put(6.3,6.3){\color{red}{2}}
            \put(7.3,6.3){na 8 je uložený ukazateľ, 2 porovnáva so 7}
        
        \pause
        
            \put(0,4){\line(1,0){7}}
            \put(0,4){\line(0,-1){1}}
            \put(1,4){\line(0,-1){1}}
            \put(2,4){\line(0,-1){1}}
            \put(3,4){\line(0,-1){1}}
            \put(4,4){\line(0,-1){1}}
            \put(5,4){\line(0,-1){1}}
            \put(6,4){\line(0,-1){1}}
            \put(7,4){\line(0,-1){1}}
            \put(0,3){\line(1,0){7}}
            
            \put(6.5,2){\line(0,1){1}}
            \put(2.5,2){\line(1,0){4}}
            \put(2.5,2){\line(0,1){1}}
            
            \put(0.3,3.3){\color{blue}{8}}
            \put(1.3,3.3){7}
            \put(2.3,3.3){6}
            \put(3.3,3.3){1}
            \put(4.3,3.3){0}
            \put(5.3,3.3){9}
            \put(6.3,3.3){\color{red}{2}}
            \put(7.3,3.3){na 8 je uložený ukazateľ, 2 porovnáva so 6}
            
        \pause

            \put(0,1){\line(1,0){7}}
            \put(0,1){\line(0,-1){1}}
            \put(1,1){\line(0,-1){1}}
            \put(2,1){\line(0,-1){1}}
            \put(3,1){\line(0,-1){1}}
            \put(4,1){\line(0,-1){1}}
            \put(5,1){\line(0,-1){1}}
            \put(6,1){\line(0,-1){1}}
            \put(7,1){\line(0,-1){1}}
            \put(0,0){\line(1,0){7}}
            
            \put(6.5,-1){\line(0,1){1}}
            \put(3.5,-1){\line(1,0){3}}
            \put(3.5,-1){\line(0,1){1}}
            
            \put(0.3,0.3){\color{blue}{8}}
            \put(1.3,0.3){7}
            \put(2.3,0.3){6}
            \put(3.3,0.3){1}
            \put(4.3,0.3){0}
            \put(5.3,0.3){9}
            \put(6.3,0.3){\color{red}{2}}
            \put(7.3,0.3){na 8 je uložený ukazateľ, 2 porovnáva so 6}
            \put(7,-1){1 je menšia ako 2 a zároveň máme ukazateľ na 8}
            
        \pause
            
            \put(0,-2){\line(1,0){7}}
            \put(0,-2){\line(0,-1){1}}
            \put(1,-2){\line(0,-1){1}}
            \put(2,-2){\line(0,-1){1}}
            \put(3,-2){\line(0,-1){1}}
            \put(4,-2){\line(0,-1){1}}
            \put(5,-2){\line(0,-1){1}}
            \put(6,-2){\line(0,-1){1}}
            \put(7,-2){\line(0,-1){1}}
            \put(0,-3){\line(1,0){7}}
            
            \put(0.5,-4){\line(0,1){1}}
            \put(0.5,-4){\line(1,0){3}}
            \put(3.5,-4){\line(0,1){1}}
            
            \put(0.3,-2.7){\color{blue}{1}}
            \put(1.3,-2.7){7}
            \put(2.3,-2.7){6}
            \put(3.3,-2.7){\color{blue}{8}}
            \put(4.3,-2.7){0}
            \put(5.3,-2.7){9}
            \put(6.3,-2.7){\color{red}{2}}
            \put(7.3,-2.7){1 a 8 sa prehodia}
            \end{picture}
        \end{figure}
\end{frame}

\begin{frame}{Quicksort pseudo kód}
    \begin{algorithm}[H]
        \caption{Quicksort rekurzívna varianta}
            \SetKwProg{Fn}{Function quicksort}{:}{}
                \Fn{($arr[]$, $low$, $high$)}{
                    \uIf{($low < high$)} {
                        pi = partition(arr, low, high)\;
                        quickSort(arr, low, pi - 1)\;
                        quickSort(arr, pi + 1, high)\;
                    }
                }
         
    \end{algorithm}
\end{frame}

\begin{frame}{Partition pseudo kód}
    \begin{algorithm}[H]
            \SetKwProg{Fn}{Function partition}{:}{}
                \Fn{($arr[]$, $low$, $high$)}{
                    pivot = arr[high]\;
                    i = (low - 1) \;
                    \For{$j = low; j <= high- 1; j++$} {
                        \uIf{($arr[j] < pivot$)} {
                            i++ \;
                            swap arr[i] and arr[j] \;
                        }
                    }
                    swap arr[i+1] and arr[high] \;
                    \Return {(i+1)}
                }
        \end{algorithm}
\end{frame}

\begin{frame}{Použité zdroje}
\begin{itemize}
    \item \texttt{\url{https://sk.wikipedia.org/wiki/Quicksort}}
    \item \texttt{\url{https://www.geeksforgeeks.org/quick-sort/}}
    \item \texttt{\url{https://www.programiz.com/dsa/quick-sort}}
\end{itemize} 
    
\end{frame}

\end{document}